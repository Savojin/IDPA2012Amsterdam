\subsection{Risikokarte}\footnote{Quelle Risikokarte: \cite{risikokarte}} 
Test 1 zu 1 von quelle:
\newline\newline
\textnormal{Die Risikokarte enthält Informationen über die Hochwassergefahr in den Niederlanden. 
Die Karte zeigt, wie hoch das Wasser in Ihrem Wohnort steigen kann und welche höher gelegenen Gebiete sicher sind.
}

\textnormal{\newline\bfseries Sie können sich auf das Hochwasser vorbereiten. 
Folgendes sollten Sie bereithalten:}
 
\begin{itemize}  
\item Ein Notpaket
\item Adressen, die Sie im Notfall aufsuchen können
\item Die Anfahrtsrouten zu diesen Adressen
\end{itemize}  

\textnormal{\newline\bfseries Wenn Hochwasser droht}
\newline
\textnormal{Sie werden mit Hilfe von lokalen Rundfunk- und Fernsehsendern, 
Alarmsirenen, SMS-Mitteilungen oder über Lautsprecherwagen der Polizei
 oder Feuerwehr gewarnt. Gehen Sie dann wie folgt vor:}
\begin{itemize}  
\item Schalten Sie den Radiosender für den Katastrophenfall ein und informieren Sie sich, falls möglich, 
auf der Website Ihrer Kommune oder unter www.crisis.nl über die aktuellen Entwicklungen.
\item Befolgen Sie die Anweisungen der Behörden und Einsatzdienste.
\end{itemize}  
\textnormal{\newline\bfseries Wenn Sie Ihre Wohnung verlassen müssen}
\begin{itemize}  
 \item   Sperren Sie Gas, Wasser und Strom ab.
\item    Nehmen Sie nur das Allerwichtigste mit (Bargeld, Medikamente und Kopien von Ausweisen und Versicherungspolicen).
\item    Schließen Sie die Wohnungstür ab.
 \item   Legen Sie Sandsäcke vor Türen und Fenster.
 \item   Vergewissern Sie sich, dass Ihre Nachbarn von der Evakuierung wissen.
\item    Tragen Sie festes Schuhwerk und wasserdichte Kleidung.
\item    Wenn Sie das Auto nehmen, legen Sie Sandsäcke vor Türen und Fenster Ihres Hauses und nehmen Sie Folgendes mit:
\item    Ihr Notpaket
\item    Ihre Haustiere und Futter/Wasser für die Tiere
\item    Einen Campingkocher mit zusätzlichem Brennstoff (Gas, Benzin), Töpfe und Essgeschirr
 \item   Eine Trillerpfeife (um Hilfe zu rufen)
 \item   Ein Tau und eine Plastikplane, mit der Sie einen Unterschlupf bauen können
 \item   Reservekraftstoff für Ihr Auto
\end{itemize}  
\textnormal{\newline\bfseries Wenn Sie die Wohnung nicht verlassen können}
\begin{itemize}  
   \item Begeben Sie sich mit Ihrem Notpaket an den höchsten Punkt in Ihrem Haus.
   \item Hängen Sie ein weißes Laken aus dem Fenster, um den Rettungskräften zu signalisieren, dass sich noch Personen im Haus befinden.
  \item  Helfen Sie anderen – wenn hierfür Zeit ist – bei den Notmaßnahmen und bieten Sie anderen sicheren Unterschlupf bei Ihnen an, wenn dies möglich ist.
\end{itemize}  
\textnormal{\newline\bfseries Während des Hochwassers}
\begin{itemize}  
  \item  Schalten Sie den Radiosender für Katastrophenfälle ein.
   \item Befolgen Sie alle Anweisungen.
  \item  Begeben Sie sich auf eine sichere Anhöhe und bleiben Sie dort. Warten Sie auf Hilfe.
 \item   Wenn für Ihre eigene Sicherheit gesorgt ist, helfen Sie Menschen in Ihrer Umgebung, die Hilfe benötigen.
  \item  Waten oder fahren Sie nicht durch das Wasser. Schnell fließendes Wasser, das mehr als knöcheltief ist, kann Sie leicht erfassen und mit sich reißen.
\end{itemize}   
\textnormal{Die folgende Übersicht zeigt, was Sie bei welchen 
Wasserständen noch tun können.}
\newline
\textnormal{\newline\bfseries Wasserstande Was können Sie selbst tun?}
\newline
\begin{tabular}[]{l p{10cm}}
0 - 20 cm &
Bringen Sie wichtige Gegenstände an einem hohen und trockenen Ort in Sicherheit. 
\newline Schützen Sie Ihren Besitz vor Schäden (Sandsäcke).
\newline Autos können noch im Schritttempo fahren. 
\newline\\ 
20 - 50 cm &
Bringen Sie sich selbst, Ihre Familie und wichtige Gegenstände in Sicherheit.
\newline Personen, die Hilfe benötigen, können noch zu Fuß erreicht werden. Helfen Sie anderen Menschen, soweit es geht.
\newline\\
50 - 80 cm &
Militärfahrzeuge können noch fahren. Rettungskräfte können noch zu Ihnen gelangen.
\newline Bringen Sie sich selbst und Ihre Familie in Sicherheit.
\newline Prüfen Sie, ob Sie an einem sicheren Ort noch anderen helfen können.
\newline\\
80 cm - 2 m &
Im ersten Stock Ihres Hauses sind Sie in Sicherheit.
\newline Bringen Sie sich selbst und Ihre Familie in Sicherheit und nehmen Sie Ihre Notvorräte sowie das Radio und Batterien mit.
\newline Schalten Sie den Radiosender für Katastrophenfälle (Lokalsender) ein und befolgen Sie die Anweisungen der Rettungskräfte.
\newline\\
2 - 5 m &
Im zweiten Stock Ihres Hauses sind Sie in Sicherheit.
\newline Bringen Sie sich selbst und Ihre Familie in Sicherheit und nehmen Sie Ihre Notvorräte sowie das Radio und Batterien mit.
\newline Schalten Sie den rampenzender [Radiosender für Katastrophenfälle] ein und befolgen Sie die Anweisungen der Rettungskräfte.
\newline\\
über 5 m &
Begeben Sie sich mit Ihren Notvorräten an den höchsten Punkt in Ihrem Haus.
\newline Halten Sie sich in der Nähe eines Ausgangs oder auf dem Dach auf, so dass Sie für die Rettungskräfte (Boot oder Hubschrauber) erreichbar sind. 
\newline Hängen Sie ein weißes Laken aus dem Fenster, um den Rettungskräften zu signalisieren, dass sich noch Personen im Haus befinden.
\\
\end{tabular}
\newline\newline 
\textnormal{\newline\bfseries Nach dem Hochwasser}
\newline 
Sie werden informiert, wenn die Gefahr in Ihrer Wohngegend vorüber ist. Begeben Sie sich nicht auf eigene Faust dorthin.
\newline\newline

