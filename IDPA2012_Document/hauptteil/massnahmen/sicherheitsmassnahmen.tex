\section{Sicherheitsmassnahmen}  
Test 1 zu 1 von quelle:
\newline\newline
ie Behörden aller Ebenen – von den Kommunen über die Provinzen bis 
hin zum Staat – setzen sich gemeinsam für den Hochwasserschutz ein.
 Das niederländische Amt für Wasserwirtschaft (Rijkswaterstaat) und
  die Wasserwirtschaftsverbände warten und kontrollieren gemeinsam 
  Deiche, Dämme und Dünen. Für die Flüsse wird zusätzlicher Raum 
  geschaffen. Die Kommunen informieren und warnen die Bevölkerung.
   Gemeinsam mit den Einsatzdiensten (Polizei, Feuerwehr und 
   medizinische Rettungsdienste) üben die Behörden regelmäßig
    den Ernstfall. Die Pegelstände werden ununterbrochen überwacht, 
    so dass im Prinzip genügend Zeit bleibt, Maßnahmen zu ergreifen
     und die Bevölkerung zu warnen.    

 