\part{Hauptteil}

Der Hauptteil ist das Herzstück jeder schriftlichen IDPA.

\section{Gliederung}
Die Gliederung des Hauptteils in einzelne Kapitel und Unterkapitel ist wichtig für die Orientierung und Verständlichkeit. Unpräzise Titel stiften Verwirrung. Die Überschriften geben die Inhalte zutreffend wieder. So entsteht ein 'roter Faden' gemäss Fragestellung.
Eine blosse Aneinanderreihung von Gesichtspunkten, die mit dem Thema zu tun haben, reicht nicht aus. Auch wenn Sie Fachliteratur zusammenfassen, so müssen Sie sich beim Formulieren jederzeit bewusst sein, inwieweit das Geschriebene Ihre in der Einleitung festgehaltene Hypothese (Fragestellung) unterstützt. Behauptungen sind erklärt, Vermutungen mit Daten, Quellen oder Querverweisen auf andere Kapitel abgestützt. Damit werden eine Argumentationsstruktur sicht- und ein logischer Aufbau erkennbar. Ihre Eigenständigkeit zeigt sich nicht nur im Satzbau (Sprachstruktur), sondern vor allem im Ziehen von eigenen Schlussfolgerungen und der Auswertung und Kommentierung von Quellen, also in der Erarbeitung einer eigenen und begründeten Aussage.
Sofern Sie in einer naturwissenschaftlich ausgerichteten Arbeit eigene Versuche durchführen (Messreihen), äussern Sie sich zuerst über das Material und die Methoden (Vorgehen), präsentieren und erklären anschliessend die Resultate. Schliesslich diskutieren und vergleichen Sie Ihre Resultate und stellen diese der bestehenden Literatur gegenüber, so dass Schlussfolgerungen gezogen werden können. 

Bilder, Tabellen, Grafiken
Bilder, Tabellen, Grafiken etc. sind erwünscht, sofern sie dem besseren Verständnis dienen. Auf Illustrationen muss im Text ausdrücklich Bezug genommen werden, d.h. die Aussage der Abbildung oder Tabelle soll kommentiert werden. Tabellen und Abbildungen sind betitelt, wobei die aussagekräftigen Titel oberhalb von Tabellen, aber unterhalb von Abbildungen stehen. Bilder, Tabellen etc. werden fortlaufend nummeriert, z.B. Abb.1, Abb.2 oder Tab.1, Tab.2 etc. Alle Illustrationen sind im Abbildungsverzeichnis aufzuführen. Bei Grafiken sind die Achsen deutlich beschriftet, die Achseneinheit sinnvoll gewählt. 

\section{Sprache}
Streben Sie eine präzise Sprache an. Schreiben Sie kurze Sätze. Nie mehr als eine Aussage soll in einem Satz Platz finden. Lange Sätze behindern das Lesen. Vermeiden Sie nach Möglichkeit Passivsätze, die unpersönliche Man-Konstruktion und den Konjunktiv. Die Wir-Form ist eine Möglichkeit, doch auch hier soll sachlich formuliert werden. Während des Schreibprozesses ist es wichtig, dass Sie klar trennen zwischen eigenen Gedanken und solchen, die Sie der Fachliteratur entnommen haben. Der Fachliteratur entnommene Gedanken kennzeichnen Sie mit einem Hinweis (z.B. Fussnote).
Zur Lösung des Problems der Gleichstellung von Frauen und Männern in der Sprache verwenden Sie eine konsequente Lösung. Statt SchülerInnen oder man/frau schreiben Sie lieber Schülerinnen und Schüler. Zu Beginn Ihrer Arbeit können Sie auch festhalten, dass mit der einen verwendeten Form auch immer die andere Form gemeint ist.

\section{Zitat}

\subsection{Beispiel 1}

Was sind Zitate und wie werden sie in den Text eingebaut? Manchmal haben schon andere Personen formuliert, was Ihnen für Ihre Arbeit wichtig erscheint. Wir unterscheiden direkte und indirekte Zitate. 

\section{Fussnoten}

Fussnoten, in kleinerer Schrift als der Schriftsatz des Textes oder einer Fusszeile, stehen in der Regel am Seitenende und erfüllen vor allem folgende drei Aufgaben:


\begin{enumerate}
\item Sie dienen der Erklärung komplizierter Begriffe und entlasten damit den Text 
\item Sie geben einen Hinweis auf andere Kapitel der Arbeit 
\item Sie machen exakte Angaben zu benutzten Quellen der Arbeit
\end{enumerate}

Studieren Sie die Darstellung von Fussnoten unter 'Gestaltung'.

\section{Gestaltung}
Vergessen Sie die Textgestaltung nicht. Trotz verlockender Möglichkeiten bleibt aber der Inhalt das Zentrale Ihrer Arbeit.
Wir empfehlen, nur eine Schrift (Arial, Helvetica) zu verwenden. Benutzen Sie die Schrifthöhe und die Schriftart (kursiv, fett) als gestalterische Mittel; in seltenen Fällen noch die Schriftfarbe. Vermeiden Sie Unterstreichungen und formatieren Sie Ihren Text linksbündig (Flattersatz) oder im Blocksatz. In beiden Fällen ist das automatische Trennungsprogramm zu benutzen. Die Schrifthöhe des Textes beträgt 11 Punkte, der Zeilenabstand 1.5, allenfalls exakt 16 Punkte. Längere direkte (wörtliche) Zitate werden mit einem Zeilenabstand von 1.0 abgehoben, kurze in den Text eingeflochten.
Kapitel sind meistens in mehrere Absätze gegliedert. Diese werden durch eine Leerzeile oder durch das Einrücken der ersten Textzeile voneinander abgehoben.

\section{Umfang}
Bei einer Zweierarbeit beträgt der Umfang des Hauptteils ohne Tabellen, Bilder, Grafiken etc. mindestens 4000 Wörter (10 A-Seiten), bei einer Dreiergruppe 6000 Wörter (15 A-Seiten). Die Ihr Projekt betreuende Lehrperson kann eine Obergrenze festlegen.
 
 