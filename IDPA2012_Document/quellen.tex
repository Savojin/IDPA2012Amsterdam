\part{Quellen}



\emph{Elektronische Quellen:}
\\ \textsc{Amsterdam} 
\\ \url{http://tanamatales.com/wp-content/uploads/2011/12/amsterdam-foto.jpg}
\\(Abrufdatum: 10.09.2012)
\\[0.4cm]
In diesem Verzeichnis halten Sie alle benutzten Quellen fest. Diese sind nach der Gattung (nicht nach den Kapiteln des Hauptteils) zu gruppieren. Innerhalb der Gattung sind die Angaben alphabetisch zu ordnen. Es lohnt sich schon zu Beginn der Arbeit, die Angaben zu den einzelnen Quellen zu notieren, damit die Zusammenstellung nicht allzu aufwändig wird. 

Grundsätzlich können folgende Gattungen unterschieden werden: gedruckte, elektronische, audiovisuelle und mündliche Quellen. 

Bei den gedruckten Quellen kann je nach Thema eine Unterscheidung in Primär- und Sekundärquellen Sinn machen. Beispiel: Der Roman eines Autors ist eine Primärquelle, das Buch eines Journalisten über den gleichen Roman eine Sekundärquelle. Bei anderen Themen mag eine Unterteilung nach Büchern und Zeitungen und Zeitschriften angezeigt sein. 

Bei einer elektronischen Quelle muss in jedem Fall das Konsultationsdatum der Quelle angegeben sein. 

Bei einer audiovisuellen Quelle sind neben dem Titel auch der Regisseur, das Erscheinungsjahr der Sendung, der Verlag (bei DVD-Produktion) oder der produzierende TV-Sender wichtig. 

Bei den mündlichen Quellen sollte die Adresse und Funktion des Gesprächspartners sowie das Gesprächsdatum festgehalten sein. Sofern die Anonymität des Gesprächspartners berücksichtigt werden muss, halten Sie dies in der Einleitung der Arbeit fest, führen aber die mündliche Quelle als anonym trotzdem auf. 

Die in der Arbeit verwendeten Abbildungen (Tabellen, grafische Darstellungen und Bilder) unterteilen Sie in ein Tabellen- und Abbildungsverzeichnis. 

Beispiel eines Quellen-, Literatur- und Abbildungsverzeichnisses.
