\part{Quellen}
Temporäre Übersicht


alte Quellen  / siehe Herbstdossier
\url{http://en.wikipedia.org/wiki/Delta_Works}
\url{http://www.wired.com/science/planetearth/magazine/17-01/ff_dutch_delta}
\url{http://risicokaart.nl/de/informatie_over_risicos/overstroming}
\url{http://www.abipur.de/referate/stat/662450251.html}
\url{http://www.youtube.com/watch?v=xqU__VgwriE} (Film)


neue Quelle:

noch nicht zugeordnet/allgemein:
\url{https://www.tu-braunschweig.de/Medien-DB/hyku-xr/xtremrisk2012_risikokonzepte_hochwasserschutz_nl.pdf}
\url{http://www.nachhaltigkeit.org/201002233797/natur-landwirtschaft/hintergrund/niederlande-wollen-vorangehen}
\url{http://bildungsserver.hamburg.de/contentblob/3365390/data/2012-meeresspiegel-bangladesch-niederlande.pdf} ursache folgen
\url{http://www.uni-muenster.de/NiederlandeNet/nl-wissen/geographie/vertiefung/wasser/index.html}



Multi-Layer \url{http://www.oranjewoud.nl/sites/default/oranjewoud_files/3_32556.pdf}

 Evakuierung
      \url{ http://www.spiegel.de/panorama/deich-undicht-niederlande-in-not-a-807633.html}
       \url{http://www.uni-muenster.de/NiederlandeNet/aktuelles/archiv/2012/januar/0106hochwasser.shtml}

       Polder abschottung à in dijkdoorbraak

       Wurten \url{http://de.wikipedia.org/wiki/Sturmflut_1962}

       Schwimmende häuser \url{http://www.spiegel.de/wissenschaft/technik/klimawandel-in-holland-wohnen-in-ebbe-und-flut-haeusern-a-800897.html}


\bibliography{quellen}
\bibliographystyle{plain}