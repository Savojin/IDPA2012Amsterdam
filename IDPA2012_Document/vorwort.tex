\part{Vorwort}



Wir haben uns zur Zusammenarbeit entschieden, da wir ähnliche Interessen verfolgen.
Zusätzlich besitzen wir alle eine ähnliche Arbeitseinstellung. 
Wir haben auch schon in der Vergangen Projekte zusammen realisiert und dabei bemerkt, dass wir ein gutes Team bilden.
 
Die Idee, unsere Arbeit den Dämmen von Holland zu witmen, kam uns sehr schnell, da wir Holland unter anderem mit Dämmen assozieren.
Da wir alle technisch interessiert sind und der technische Aspeckt der Dämme recht umfassend ist, boten sich die Dämme als gutes Thema an. 
Desweitern haben Naturkatastrophen einen faszinierenden Effekt und eignen sich gut zur Recherche.
Auch andere Bereiche dieses Themas wie Wirtschaft, Politik und Gesellschaft lassen sich gut in unser Projekt einbinden. 
Um den Gesellschaftlichen Aspekt abzudeken, bittet es sich zusätzlich an Einwohner von Amsterdam zu befragen.

Der Technische Aspekt der Dämme reichte uns als begeisterte Informatiker nicht. Deshalb entschieden wir uns, die Indiziplinäre 
Projektarbeit nicht mit dem Standart Word Programm zu schreiben. Claudia Saxer brachte den Vorschlag, die Arbeit mit dem Textverarbeitungsprogramm
LaTex zu schreiben. 
LaTex eignet sich gut für die einheitliche Formatierung der Dokument und hat uns schon beim ersten Anblick fasziniert, da es uns an die 
Programmierung  von Websites erinnert.


TODO
Danksagung
