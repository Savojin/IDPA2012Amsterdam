\part{Reflexion}

\section{Stefan Kull}
Als Gruppe waren wir dazu in der Lage gut zusammen zu arbeiten, da wir 
ähnliche Vorstellungen haben und uns gut verstehen. Gut war weiterhin, dass niemand 
versucht hat die anderen herumzukommandieren. 
Probleme bei der Zusammenarbeit waren hauptsächlich fehlende Motivation und daraus 
entstehendes Chaos beim Zeitmanagement.
Gelernt habe ich einiges über Holland. Zudem ist mir klar geworden, was für Chaos ohne genaues
Zeitmanagement entsteht.

\section{Simon Schneider}
Wir vertreten alle die selbe Arbeitseinstellung, was zu einer guten Zusammenarbeit innerhalb
der Gruppe führte.
Es wäre von Vorteil gewesen, hätte jemand für eine strukturierte Zeitplanung gesorgt.
Ich bin jedoch froh, dass sich keiner von sich aus zum Boss hervorgehoben hat.
Bei größeren Entscheidungen schauten wir immer, dass alle Mitglieder der Gruppe einverstanden
waren. Somit konnten wir Konflikte verhindern. 
Trotz der fehlenden Motivation konnte ich darauf vertrauen, dass alle den nötigen Einsatz
brachten.
Ein weiteres Mal, wies mich mein Leben darauf hin, ich sollte mehr Planen und ein 
weiteres Mal habe ich den Rat ignoriert. 

Ich habe gelernt, dass man durch ein gescheites Zeitmanagement viel Zeit sparen kann.
Abgesehen davon habe ich viel neues über Holland gelernt. 

\section{Claudia Saxer}
Das ähnliche Arbeitsverhalten und die ähnlichen Meinungen unter den Gruppenmitglieder sorgte
für ein gutes Arbeitsklima und eine gute Zusammenarbeit. Da wir uns gut verstehen und uns 
oft untereinander absprechen, gibt es fast keine Meinungsverschiedenheiten. 
Leider fehlte mir manchmal die nötige Motivation, welches zur Vernachlässigung der Arbeit 
führte. 
Eine Herausforderung war das Einrichten unserer {LaTeX} Umgebung, ich scheitere immer noch
am einbinden der Rechtschreibkontrolle.

Ich habe während der letzten paar Monate einiges über {LaTeX} gelernt.
Auch wurde mir bewusst wie hilfreich ein gutes Zeitmanagement ist. 
Wie die anderen habe auch ich einiges über Holland gelernt.