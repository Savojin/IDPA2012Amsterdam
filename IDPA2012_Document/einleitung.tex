\part{Einleitung}

Wir haben uns zur Zusammenarbeit entschieden, da wir ähnliche Interessen verfolgen.
Zusätzlich besitzen wir alle eine ähnliche Arbeitseinstellung. 
Wir haben auch schon in der Vergangenheit Projekte zusammen realisiert und dabei bemerkt, dass wir ein gutes Team bilden.
 
Die Idee, unsere Arbeit den Dämmen von Holland zu widmen, kam uns sehr schnell, da wir Holland unter anderem mit Dämmen assoziieren.
Da wir alle technisch interessiert sind und der technische Aspekt der Dämme recht umfassend ist, boten sich die Dämme als gutes Thema an. 
Des Weiteren haben Naturkatastrophen einen faszinierenden Effekt und eignen sich gut zur Recherche.
Auch andere Bereiche dieses Themas wie Wirtschaft, Politik und Gesellschaft lassen sich gut in unser Projekt einbinden. 
Um den Gesellschaftlichen Aspekt abzudecken, bittet es sich zusätzlich an Einwohner von Amsterdam zu befragen.

Der Technische Aspekt der Dämme reichte uns als begeisterte Informatiker nicht. Deshalb entschieden wir uns, die Interdisziplinäre 
Projektarbeit nicht mit dem Standard Word Programm zu schreiben. Claudia Saxer brachte den Vorschlag, die Arbeit mit dem Textverarbeitungsprogramm
{/LaTeX} zu schreiben. 
LaTex eignet sich gut für die einheitliche Formatierung der Dokument und hat uns schon beim ersten Anblick fasziniert, da es uns an die 
Programmierung  von Websites erinnert.


\part{Projektskizze}
 
Den Schwerpunkt bei unserer Arbeit wollen wir auf die Auswirkungen 
einer Überflutung und die „Wiederaufbaustrategie“ der Niederlande setzen. 
Da es zu diesen Themen jedoch anscheinend nicht genügend Quellen vorhanden sind, 
um die ganze Arbeit darauf aufzubauen, wollen wir uns zusätzlich noch intensiv mit 
den Plänen der Niederlande für die Zukunft beschäftigen. Auch interessieren wir uns 
sehr dafür, wie gut die Bevölkerung z.B. über Notfallpläne informiert wird. 
Weitere Themen die wir anschauen wollen bzw. für die Schwerpunktthemen von 
Bedeutung sind wären z.B.
 
\begin{itemize}  
\item Bestehende Schutzmaßnahmen
\item Vergangene Überflutungen
\item Meinung der Bevölkerung / Einschätzung der Gefahr
\item Informieren der Bevölkerung über zukünftige Gefahren
\end{itemize}    
