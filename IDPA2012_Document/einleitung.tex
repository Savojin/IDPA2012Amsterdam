\part{Einleitung}
In der Einleitung halten Sie präzise fest, was sie untersuchen möchten. Sie grenzen das Thema (Untersuchungsgegenstand) Ihrer Arbeit ein und ordnen es in einen fachlich übergeordneten Zusammenhang ein. 
Sie nennen und begründen die Problemstellung und die Ziele. Dazu formulieren Sie eine oder mehrere Thesen* (Fragestellung) und begründen sie. Auch die zur Zielerreichung angewandten Methoden und der Zugang und die Qualität von Informationen werden thematisiert. Der Aufbau des Hauptteils und der Zusammenhang einzelner Hauptkapitel zueinander werden erklärt.

*Eine These ist zunächst eine unbewiesene Annahme, eine gehaltvolle Behauptung von Gesetzlichkeiten oder Tatsachen. Sie will die vermuteten Gesetzlichkeiten oder Tatsachen beweisen oder widerlegen. Die These ist ein Hilfsmittel für die Gewinnung von wissenschaftlichen Erkenntnissen, ein Vorentwurf für eine Theorie. 

Hinweis: 

Zu Beginn der schriftlichen Arbeit dient die Projektskizze als Einleitung. Die exakte Formulierung der Einleitung mit den oben erwähnten Punkten erfolgt während oder am Schluss der Arbeit. Der Umfang der Einleitung sollte mindestens 300 und höchstens 600 Wörter umfassen.
